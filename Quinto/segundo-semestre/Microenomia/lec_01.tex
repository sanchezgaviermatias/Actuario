\lecture{1}{mié 01 sep 2021 09:57}{Guia 2}

\section{Guía 2}
    
Estrategia Razionalizable, en algún momento es la mejor respuesta.

\subsection{ Ejercicio 4 } 
\begin{enumerate}
    \item 
        Toma examen por curva de nivel.  $e_i$ es el esfuerzo del jugador  $i$.

        \begin{align*}
            s_i &= e_i \in  [0;\; 10] \\
            U_i &= X_i - e_i \\
        .\end{align*}

        Los empates nunca serán equilibrios ya qeu hay incentivos para ambos a cambiar.
        Tampoco hay equilibro de nash en puras.\\[0.2cm]
        Todas las estrategias son {\color{blue} razionalizables}. \\[0.2cm] 
        Hacer $10$ de esfuerzo es debilemnte dominada, siempre prefiero hacer 0?.

    \item   este lo haces vos crack.
    
\end{enumerate}

\subsection{ Ejercicio 5 } 

$n$ personas. $ X_i  \in  [0 ;\;1] $ 

\[
U_i = -  \(\lvert x_i -  \sum_{j=1}^{n} \frac{x_j}{n}  \rvert\) 
.\] 

Lo mejor que puedo hacer es tirar promedio simple, es el qeu maximiza la utilidad . 
El equilibrio de Nash se da cuando todos los jugadores juegan lo mismo. \\[0.2cm] 

La función de mejor respuesta

\[
x_i^{\text{ Mejor Respuesta }  }  =  \text{ Promedio general de los compas } 
.\] 

\subsection{ Ejecicio 10 } 

\begin{align*}
    f_1(r_1)	 &=  8 r_1  - \frac{r_1^{2}}{2}\\
    f_2(r_2) &=  4r_2  \\
.\end{align*}

