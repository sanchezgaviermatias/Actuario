\lecture{2}{mar 07 sep 2021 08:09}{Tercera clase de Contabilidad }


\section{Guía Práctica}

\begin{enumerate}
	\item  El ejercicio es de año calendario, porqeu se compara de año a año. \\[0.2cm] 

		El año calendario es comparar año a año, en Argentina es junio a junio.

		\[
			\text{ Prima Ganada }  = \text{ Prima Emitida }  - \Delta \text{ Variación de Riesgo en Curso } 
		.\] 	

		Osea la prima ganada, es la prima emitida del año menos la reserva que transcurrio en el año.  

	\item   No lo hizo

	\item  Este ejercicio es muy contable 

	\item \textbf{Emitidos por año Calendario}  Suponemos que la fecha de Emisión coincide con la fecha de Vigencia (supuesto, por falta de info).

	Hay que chequear la periocidad, en este caso son todas polizas semestrales. Por ende los \textbf{expuestos son $0.5$}		
	
	El D:  hay una cancelación (la expreso negativo), cuanto mesess le falta dar cobertura?  $-0,25$

	\textbf{Underwriting year}, solo importa el año donde se corrio el riesgo, cuanto ganas,  (inicio de vigencia)
	

\end{enumerate}

