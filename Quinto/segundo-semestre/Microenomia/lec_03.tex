\lecture{3}{lun 06 sep 2021 10:01}{Subastas Parte 2}


\section{Valuación de Primer precio}
Cuando se toma el bid de mayor valor.	

\textbf{Equilibrio de Nash:} en subasta de primer precio siempre gana el jugador que más lo valora. Esto no sucede en Subastas de segundo precio, en esta puede haber casos donde no gana el que más valora. ahre loco

\textbf{Conclusiones: } 
\begin{itemize}
	\item En subasta de primer precio, apostar tu valuación es debilmente dominada (creo que para todos, no se si el top valuador)
	\item   En el equilibrio de Nash, por lo menos hay alguíen que juega una estrategia
		debilmente dominada.
\end{itemize}

\begin{align*}
	(v_1;\; v_2;\; v_3 ;\; v_4\ldots ;\; v_n)  &\to  \text{ 2do precio } \\
	(v_1;\; v_2;\; b_3 ;\; b_4 \ldots ;\; b_n)  &\to  \text{1er precio} 
.\end{align*}

En el primer precio da lo mismo lo que haga el resto, solo importa los dos primeros. En segundo precio
todos valuan lo mismo.  \\[0.2cm] 

\textbf{Super Teorema:} dado ciertos supuestos, da lo mismo si la subasta es de primer precio o segundo precio, el resultado final es el mismo.


Mirar Bertrand, y comparar con subasta de segundo precio.






