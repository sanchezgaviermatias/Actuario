\lecture{1}{dom 31 ago 2021 09:00}{Seguros Personales}

\section{Factor Biómetrico}
 
Es el beneficio por aber sobrevivido.

\[
    i (x;\; 0;\; t) =  \frac{l(x)}{l(x+t)}

.\] 


Hay tres variables qeu definen esta situación, {\color{blue} tiempo} ,  {\color{red}  
Beneficio individual}, dependiendo la incognita podemos obtener la otra.

\subsection{Descuento Biómetrico } 

\[
    \begin{align*}
        1- d_b(x;\; 0;\; t) &= \frac{l(x+t)}{l(x)} \\
        d_b(x;\; 0;\; t) &= q(x;0;t)
    .\end{align*}
.\] 



\begin{table}[htpb]
    \centering
    \caption{caption}
    \label{tab:label}
    \begin{tabular}{ccc}
 🤯   & capitalización  &  Actualización \\[0.2cm] 
   Tasa & $p^{-1}(x;t)-1$  &  $q(x;0;t)$ \\[0.2cm] 
   factor & $p^{-1}(x;t)$  &  $1-q(x;0;t)$ 
    \end{tabular}
\end{table}


En calculo financiero

\[
    v = \frac{1}{(1+i)}
.\] 

Por tanto en biómetrico

\[
q(x;0;t) = \frac{1}{p(x;t)^{-1}}
.\] 


\[
p(x;t)^{-1} = \frac{1}{q(x;0;t)}
.\] 

A partir de esto podemos trabajar con la la guía de equivalencias.


\section{Regimen Continuo}



(calculo financiero) Dada una tasa de interés  $i$ y de descuente $d$, ambas  {\color{red} efectivas anuales }:
\[
    \delta = \ln(1+i) = -\ln(1-d)
.\] 
    
El monto a cabo de t años, en regimen continuo 
\[
C_t = C_0 \cdot e^{t \cdot \delta}
.\] 


\subsection{ Equivalencia de Tasas }
Tasa de interés nominal $i$  de capitalización $m$ (semestral 2, cuatrimestral 3, trimestral 4, biemstral 6, mensual 12, diaria 360) :

\[
    (1+i)^{m} = (1+i)
.\] 



Luego por analogía
\[
    \begin{align*}
        \lambda  &= \ln p^{-1}(x;\;t) \\
        - \lambda &= p(x;1) \\[0.2cm] 
        p(x;1) &= e^{-\lambda}
    .\end{align*}
.\] 

recordando que 

\[
    p(x;1) = e^{\int_0^{1} \m(x + t)dt}
.\] 

entonces se deduce que 
\[
    \lambda = \int_0^{1} \mu(x+t)dt
.\] 

si $\mu$ es constante
\[
    \lambda = \mu(x)
.\] 



\subsection{ Grupo de Sobrevieintes } 

Esto es un proceso estocástico, recordar proceso de vida, muerte .











