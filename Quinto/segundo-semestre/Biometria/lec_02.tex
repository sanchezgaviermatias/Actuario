\lecture{2}{dom 29 ago 2021 01:10}{Factor Biómetrico}
\section{Factor Biómetrico}

Es el beneficio por aber sobrevivido.

\[

    i (x;  0;  t) =  \frac{l(x)}{l(x+t)}
\] 


Hay tres variables qeu definen esta situación, {\color{blue} tiempo} ,  {\color{red}
Beneficio individual}, dependiendo la incognita podemos obtener la otra.

\subsection{Descuento Biómetrico }
\[
    \begin{align*}
        1- d_b(x;  0;  t) &= \frac{l(x+t)}{l(x)}\\
        d_b(x;  0;  t) &= q(x;0;t)
    \end{align*}
\]  

\begin{table}[htpb]
    \centering
    \caption{caption}
    \label{tab:label}
    \begin{tabular}{ccc}
    & capitalización  &  Actualización
   Tasa &  p ⁻¹ (x;t)-1   &   q(x;0;t) 
   factor &  p ⁻¹ (x;t)   &   1-q(x;0;t) 
    \end{tabular}
\end{table}


En calculo financiero
\[
    v = \frac{1}{(1+i)}
\] 
 

Por tanto en biómetrico


