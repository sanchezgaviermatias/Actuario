\documentclass[a4paper]{article}

\usepackage[utf8]{inputenc}
\usepackage[T1]{fontenc}
\usepackage{textcomp}
\usepackage[spanish]{babel}
\usepackage{amsmath, amssymb}

%Programing
\usepackage{listings}
\usepackage{xcolor}

\definecolor{codegreen}{rgb}{0,0.6,0}
\definecolor{codegray}{rgb}{0.5,0.5,0.5}
\definecolor{codepurple}{rgb}{0.58,0,0.82}
\definecolor{backcolour}{rgb}{0.95,0.95,0.92}

\lstdefinestyle{mystyle}{
    backgroundcolor=\color{backcolour},   
    commentstyle=\color{codegreen},
    keywordstyle=\color{magenta},
    numberstyle=\tiny\color{codegray},
    stringstyle=\color{codepurple},
    basicstyle=\ttfamily\footnotesize,
    breakatwhitespace=false,         
    breaklines=true,                 
    captionpos=b,                    
    keepspaces=true,                 
    numbers=left,                    
    numbersep=5pt,                  
    showspaces=false,                
    showstringspaces=false,
    showtabs=false,                  
    tabsize=2
}



% figure support
\usepackage{import}
\usepackage{xifthen}
\pdfminorversion=7
\usepackage{pdfpages}
\usepackage{transparent}
\newcommand{\incfig}[1]{%
	\def\svgwidth{\columnwidth}
	\import{./figures/}{#1.pdf_tex}
}

\pdfsuppresswarningpagegroup=1
\lstset{style=mystyle}


\title{Variables Aleatorias}
\author{Matías Sanchez Gavier }

\begin{document}
\maketitle

\section{Introducción}
Esto es una  generlización del tema de probabilidad, las variables aleatorias son \textbf{funciones} $ X: \Omega \mapsto \mathbb{R}  $. Es decir que para cada resultado
del experimento (\mathbb{E}) hay un valor asociado que es elemento de los reales. Esto es por la definición de funciones, para cada valor
del dominio hay un valor asociado. 





\subsection{ clasificación de variables aleatorias } 
la clasificación se da por el dominio de $\omega$ (espacio muestral), cuando este conjunto es infinto y no contable (mira en wikipedia la definición) , se denomina {\color{blue} variable continua}  , por ejemplo
el tiempo de vida de una persona es continuo, en cambio, en cualquier otro caso la variable es  {\color{blue} discreta} , por ejemplo la edad de muerte  (si es que la medimos en años enteros). 


\section{variables discretas}
en la práctica hay variables aleatorias muy comúnes, con pocas variables uno es capaz de representar la gran mayoría de los problemas en el mundo (wow) . las variables aleatorias discretas que vamos a  
ver son: 

\begin{itemize}
	\item  binomial
	\item  bernoulli (caso especial de binomial )
	\item poisson 
	\item binomial negativa 
	\item pascal  (caso especial de binomial negativa )
\end{itemize}

en total son 3 distribuciones, y hay dos casos especiales.




\subsection{ Condiciones Necesarias } 










 





\end{document}
