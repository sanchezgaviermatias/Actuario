\documentclass[a4paper]{article}

\usepackage[utf8]{inputenc}
\usepackage[T1]{fontenc}
\usepackage{textcomp}
\usepackage[spanish]{babel}
\usepackage{amsmath, amssymb}
\usepackage{tcolorbox}

%Programing
\usepackage{listings}
\usepackage{xcolor}

\definecolor{codegreen}{rgb}{0,0.6,0}
\definecolor{codegray}{rgb}{0.5,0.5,0.5}
\definecolor{codepurple}{rgb}{0.58,0,0.82}
\definecolor{backcolour}{rgb}{0.95,0.95,0.92}

\lstdefinestyle{mystyle}{
    backgroundcolor=\color{backcolour},   
    commentstyle=\color{codegreen},
    keywordstyle=\color{magenta},
    numberstyle=\tiny\color{codegray},
    stringstyle=\color{codepurple},
    basicstyle=\ttfamily\footnotesize,
    breakatwhitespace=false,         
    breaklines=true,                 
    captionpos=b,                    
    keepspaces=true,                 
    numbers=left,                    
    numbersep=5pt,                  
    showspaces=false,                
    showstringspaces=false,
    showtabs=false,                  
    tabsize=2
}



% figure support
\usepackage{import}
\usepackage{xifthen}
\pdfminorversion=7
\usepackage{pdfpages}
\usepackage{transparent}
\newcommand{\incfig}[1]{%
	\def\svgwidth{\columnwidth}
	\import{./figures/}{#1.pdf_tex}
}

\pdfsuppresswarningpagegroup=1
\lstset{style=mystyle}


\title{ Clase Probabilidad}

\author{Matías Sanchez Gavier }

\begin{document}
\maketitle

\section{Probabilidad Condicionada}
	

\begin{tcolorbox}[title=Probabilidad Condicionada]
Probabilidad de $A$ dado $B$ 
\tcblower

\[
	P(A|B) = \frac{P(A \cap B)}{ P(B) }
.\] 
\end{tcolorbox}



Si dos eventos son {\color{blue} independientes}  entonces:
\begin{tcolorbox}[title=Indepencia]
\[
	P(A \cap B ) =  P(A) \cdot P(B)
.\] 
\end{tcolorbox}

Aplicando esto a la probablidad condicionada:

\begin{tcolorbox}[title=Independencia en Condicionada ]
\[
	P(A|B) = \frac{P(A \cap B)}{ B } =  \frac{P(A) \cdot P(B)}{P(B)} = P(A)
.\] 
\end{tcolorbox}



\subsection{ Bayes } 

\begin{tcolorbox}[title=Teorema de Probabilidad Total]
Partimos del  eorema de probablidad total, sea $A_i$ una partición del espacio muestral (osea $A_i \cup_{i=1}^{n} = \Omega$)
\tcblower
\[
	P(B) =  P(B|A_1) \cdot P(A_1) + P(B|A_2) \cdot P(A_2)
.\] 
\end{tcolorbox}










\end{document}
